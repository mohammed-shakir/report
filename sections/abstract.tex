\begin{abstract}
Web mapping services let users explore and query geographic datasets, for example by requesting vector features inside a map view. In many deployments, GeoServer (a server that publishes geospatial data through standard OGC web services) exposes standard web APIs on top of a PostGIS database. However, when traffic becomes concentrated to a few popular areas (urban hotspots), performance can degrade even with spatial indexes, because many requests overlap and repeatedly trigger similar database work, increasing latency and backend load. This thesis develops an approach for building high-load, interactive vector query services on top of GeoServer/PostGIS by adding an adaptive middleware cache. The approach is based on partitioning each request footprint into H3 hexagonal cells, caching per-cell results in Redis, and maintaining freshness through per-key \ac{ttl} combined with Kafka-driven invalidation. The main idea is to use spatial locality under skew: when many requests overlap the same cells, cached cell payloads can be reused and composed into an exact response while reapplying the original spatial and attribute filters. The proposed methods are implemented as a prototype Go-based middleware.

The prototype is evaluated in a containerized testbed under Zipf-like skew at 800 requests/s, comparing against a no-cache baseline using latency percentiles, throughput, and backend resource usage.

Results show that, when traffic is concentrated in a few hotspot areas, the cache makes the system about 6x faster in the slow cases and cuts PostGIS CPU by about 10x, while still keeping up with the load. The main trade-off is increased memory usage and composition overhead in the caching path, especially when requests span many cells.

\end{abstract}
\addcontentsline{toc}{section}{Abstract}
