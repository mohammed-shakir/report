\begin{abstract}
Web mapping services let users explore and query geographic datasets, for example by requesting vector features inside a map view. In many deployments, GeoServer exposes standard web APIs on top of a PostGIS database. Even with spatial indexes, performance can drop when traffic is concentrated to a few popular areas (urban "hotspots"), because many requests overlap and trigger similar database work repeatedly, increasing latency and database load. This thesis develops and evaluates an adaptive middleware cache that partitions request footprints into H3 hexagonal cells, stores per-cell results in Redis, and maintains freshness via time-to-live (\ac{ttl}) and Kafka-driven invalidation. A Go middleware is implemented that maps \ac{bbox}/polygon filters to H3 cells, composes per-cell payloads on cache hits, deduplicates features, and reapplies the exact spatial and attribute filters.

The solution is evaluated in a containerized testbed under skewed (Zipf-like) request patterns, measuring latency percentiles, throughput, and backend resource usage compared to a no-cache baseline.

Results show that the cache can substantially reduce tail latency and significantly offload PostGIS CPU under hotspot-heavy workloads, while maintaining throughput. The main trade-offs are increased memory use in the caching path and extra composition work when requests span many cells.

Overall, the results indicate that H3-guided, per-cell caching is a practical complement to \ac{gist}-indexed PostGIS for interactive vector workloads, and that combining \ac{ttl} with event-driven invalidation helps keep cached results reasonably fresh without discarding unaffected regions.
\end{abstract}
\addcontentsline{toc}{section}{Abstract}
