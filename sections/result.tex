\section{Result}
This section summarizes the main empirical outcomes of the thesis, based on the verification and validation presented in the previous chapter.

Across the 800\,\ac{rps} experiments (Zipf $s\in\{1.1,1.2,1.3,1.4\}$), the middleware cache delivers large improvements in end-to-end latency and substantially reduces database work. The strongest configuration on the testbed is H3 resolution $r{=}7$. Compared to the baseline (no cache), $r{=}7$ reduces median latency (P50) from roughly 5.1--5.9\,ms to about 0.94--1.08\,ms, and tail latency (P95/P99) from roughly 27--35\,ms / 90--100\,ms to about 4.7--6.4\,ms / 8.5--13.5\,ms (Figures~\ref{fig:latency-zipf-800} and \ref{fig:latency-speedup-zipf-800}). In parallel, PostGIS CPU is reduced by about an order of magnitude under the same offered load (e.g., $\sim$59\%\,$\rightarrow$\,6\% at $s{=}1.1$ and $\sim$56\%\,$\rightarrow$\,4\% at $s{=}1.4$; Figure~\ref{fig:postgis-cpu-zipf-800}), indicating that the gains come from avoiding repeated database evaluation in hotspot regions, not only from faster response serialization.

Resolution affects the balance between reuse and per-request overhead. At $r{=}8$, performance is mixed at low skew (heavier tails than baseline at $s{=}1.1$--$1.2$) but improves clearly as skew increases ($s{\ge}1.3$), where reuse is high enough to offset extra key and merge work (Figure~\ref{fig:latency-h3res-800-by-zipf}). At $r{=}9$, the cache is not viable at 800\,\ac{rps} on this testbed (throughput shortfall and many errors), so it is excluded from the main conclusions at this load (Figure~\ref{fig:throughput-stability-zipf-800}).

The primary trade-off observed is memory on the GeoServer side: caching increases GeoServer memory by roughly 20--40\% in the 800\,\ac{rps} scenarios (Figure~\ref{fig:geoserver-mem-zipf-800}), while PostGIS memory remains close to baseline. Overall, the results support that an H3-guided, Redis-backed middleware cache can substantially improve responsiveness and backend headroom for skewed vector workloads, with $r{=}7$ as the most robust starting point for the evaluated setup.

